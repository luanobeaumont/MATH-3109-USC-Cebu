% ==========================================
%              MAIN CONTENT
% ==========================================

Raw Data based from Kaggle: https://www.kaggle.com/datasets/uciml/pima-indians-diabetes-database/data

\section{Typography and Formatting}

\subsection{Text Sizes}
The following demonstrates the standard \LaTeX\ font sizing hierarchy:
\begin{itemize}
    \item \tiny{This is tiny text}
    \item \scriptsize{This is scriptsize text}
    \item \footnotesize{This is footnotesize text}
    \item \small{This is small text}
    \item \normalsize{This is normalsize text (Default)}
    \item \large{This is large text}
    \item \Large{This is Large text}
    \item \LARGE{This is LARGE text}
    \item \huge{This is huge text}
    \item \Huge{This is Huge text}
\end{itemize}

\subsection{Text Styles}
Standard formatting options include:
\begin{itemize}
    \item \textbf{Bold Text Example}
    \item \textit{Italicized Text Example}
    \item \underline{Underlined Text Example}
    \item \sout{Strikethrough Text Example}
    \item \textbf{\textit{Combined Bold and Italic}}
    \item \texttt{Monospaced/Typewriter Font}
    \item \textsc{Small Caps Text}
\end{itemize}

\subsection{Enumeration and Lists}
Using the \inlinecode{enumitem} package, we can customize list markers.

% Arabic Numerals
\paragraph{Arabic Numerals (Default)}
\begin{enumerate}
    \item First item (Standard)
    \item Second item
    \item Third item
\end{enumerate}

% Roman Numerals
\paragraph{Roman Numerals}
\begin{enumerate}[I.]
    \item Major Point One
    \item Major Point Two
    \begin{enumerate}[i.]
        \item Minor point i
        \item Minor point ii
    \end{enumerate}
\end{enumerate}

% Letters / Alpha
\paragraph{Alphabetical Lists}
\begin{enumerate}[A.]
    \item Section A
    \item Section B
    \begin{enumerate}[a)]
        \item subsection a
        \item subsection b
    \end{enumerate}
\end{enumerate}

\clearpage

\section{Tables}

\subsection{Standard Professional Table (No Stripes)}
\begin{table}[ht]
    \centering
    \caption{A clean academic table using Booktabs.}
    \label{tab:std_table}
    \begin{tabular}{llr} 
        \toprule
        \textbf{Component} & \textbf{Material} & \textbf{Cost (\$)} \\ 
        \midrule
        Chassis            & Aluminum          & 45.00 \\
        Microcontroller    & Silicon           & 12.50 \\
        Power Supply       & Li-Ion            & 22.00 \\
        \bottomrule
    \end{tabular}
\end{table}

\subsection{Zebra Striped Table}
This table uses the manila background color defined in the main file.

\begin{table}[ht]
    \centering
    \caption{Inventory list with alternating row colors.}
    \label{tab:zebra_table}
    \begin{NiceTabular}{l c c}
        \CodeBefore
            \rowcolors{2}{codebg}{white}
        \Body
        \toprule
        \textbf{Item Name} & \textbf{Stock ID} & \textbf{Quantity} \\ 
        \midrule
        Network Switch     & NS-001            & 15 \\
        Patch Cable (5m)   & PC-005            & 120 \\
        Server Rack        & SR-990            & 4 \\
        Cooling Unit       & CU-202            & 8 \\
        UPS Battery        & UP-110            & 20 \\
        \bottomrule
    \end{NiceTabular}
\end{table}

\subsection{Full Width Table}
This table automatically expands to fill the text width between margins.

\begin{table}[ht]
    \centering
    \caption{A full-width table utilizing available margin space.}
    \label{tab:full_width}
    % Using tabular* with extracolsep fill to push columns apart
    \setlength{\tabcolsep}{0pt}
    \begin{tabular*}{\textwidth}{@{\extracolsep{\fill}} l c r }
        \toprule
        \textbf{Project Phase} & \textbf{Status} & \textbf{Completion Date} \\
        \midrule
        Phase 1: Planning      & Complete        & Jan 2024 \\
        Phase 2: Development   & In Progress     & Mar 2024 \\
        Phase 3: Testing       & Pending         & May 2024 \\
        Phase 4: Deployment    & Pending         & Jul 2024 \\
        \bottomrule
    \end{tabular*}
\end{table}

\clearpage

\section{Figures and Images}

\subsection{Standard Figure}
\begin{figure}[H]
    \centering
    % Using example-image-a as placeholder
    \includegraphics[width=0.6\textwidth]{example-image-a}
    \captionsetup{style=mycustom}
    \caption{A standard centered figure.}
    \label{fig:standard}
    \vspace{1mm}
    {\footnotesize\textcolor{docgray}{\textit{Source: Generated by LaTeX}}}
\end{figure}

\subsection{Side-by-Side Figures}
\begin{figure}[H]
    \centering
    \begin{subfigure}[b]{0.48\textwidth}
        \centering
        \includegraphics[width=\textwidth]{example-image-b}
        \caption{Left Side Data}
        \label{fig:left}
    \end{subfigure}
    \hfill
    \begin{subfigure}[b]{0.48\textwidth}
        \centering
        \includegraphics[width=\textwidth]{example-image-c}
        \caption{Right Side Data}
        \label{fig:right}
    \end{subfigure}
    \caption{Comparative analysis showing two figures side-by-side.}
    \label{fig:sidebyside}
\end{figure}

\subsection{Custom Styled Figure}
% Specific custom style requested
\begin{figure}[H]
  \begin{center}
      \rule{\textwidth}{0.8pt}  % Overbar at top
  \end{center}
  \captionsetup{style=mycustom}
  \caption{Extended detail overview of the figure shown below...}
  \label{fig:1}
  \begin{center}
      % Note: Replace example-image with resources/2_bLine_regressplot.png
      \includegraphics[width=0.8\textwidth]{example-image}
  \end{center}

  \vspace{1mm} 
  {\footnotesize\textcolor{gray}{\textit{Source: --Source of the Graph--}}}
\end{figure}

\clearpage

\section{Code Implementation}

\subsection{Inline Code}
You can mention variables or commands directly in the paragraph using \inlinecode{inline code styling}, which highlights the text with the Manila theme.

\subsection{Multi-Language Support}

% 1. C/C++ Example
\begin{codeblock}{C}{MEMORY ALLOCATION ROUTINE}
#include <stdio.h>
#include <stdlib.h>

int main() {
    int *ptr;
    int n = 5;

    // Dynamically allocate memory
    ptr = (int*)malloc(n * sizeof(int));

    if (ptr == NULL) {
        printf("Memory not allocated.\n");
        exit(0);
    }
    
    printf("Memory successfully allocated using malloc.\n");
    free(ptr);
    return 0;
}
\end{codeblock}

\vspace{1em}

% 2. Python Example
\begin{codeblock}{Python}{DATA PROCESSING SCRIPT}
import pandas as pd
import numpy as np

def calculate_metrics(data_frame):
    # Calculate basic statistics
    mean_val = np.mean(data_frame['value'])
    max_val = np.max(data_frame['value'])
    
    print(f"Processing Complete.")
    print(f"Mean: {mean_val:.2f} | Max: {max_val}")
    return mean_val

# Execute
df = pd.DataFrame({'value': [10, 20, 30, 40, 50]})
calculate_metrics(df)
\end{codeblock}

% 3. R Example
\begin{codeblock}{R}{STATISTICAL MODELING}
# Load necessary library
library(ggplot2)

# Create sample dataset
data <- data.frame(
  x = c(1, 2, 3, 4, 5),
  y = c(2, 4, 6, 8, 10)
)

# Linear Regression Model
model <- lm(y ~ x, data = data)

# Output summary
print(summary(model))
\end{codeblock}

\subsection{System Output}
% Code Output Block
\begin{codeoutput}{TERMINAL OUTPUT}
user@workstation:~/projects$ gcc main.c -o main
user@workstation:~/projects$ ./main
Memory successfully allocated using malloc.
user@workstation:~/projects$ _
\end{codeoutput}