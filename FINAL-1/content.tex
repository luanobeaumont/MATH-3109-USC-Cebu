% ==========================================
%              MAIN CONTENT
% ==========================================

Raw Data based from Kaggle: % https://www.kaggle.com/datasets/uciml/pima-indians-diabetes-database/data


\section{Introduction}

Diabetes is a significant public health concern that demands accurate methods 
for early detection. Predictive modeling allows us to identify individuals at 
high risk before severe complications arise. This study utilizes the Pima Indians 
Diabetes Database to construct robust statistical models. We aim to determine 
which diagnostic measurements most accurately predict the onset of diabetes and 
to model the non-linear fluctuations of insulin levels using applied regression 
techniques.

\subsection{Project Objective}

This project aims to predict the onset of diabetes based on diagnostic measures. 
We will also model insulin levels using regression techniques. The goal is to 
identify key risk factors and understand non-linear relationships in the medical 
data.

\subsection{Literature Review}

Nassiwa and Zeng identify Glucose and BMI as the primary predictors of diabetes. 
This study seeks to replicate those findings using the Pima Indians dataset. We 
hypothesize that metabolic markers will outrank demographic factors like age in 
predictive power.

\subsection{Raw Data}

The dataset consists of 768 observations from the Pima Indians Diabetes Database. 
It includes medical predictor variables and one target variable, Outcome. Initial 
inspection reveals missing values encoded as zeros, which requires correction 
before analysis.

\begin{codeblock}{R}{R | SETTING UP ENVIRONMENT}
library(mice)
library(corrplot)
library(ggplot2)
library(splines)
library(performance)
library(see)
library(ggpubr)

setwd("~/School/MATH-3109/FINAL-1")
raw_data <- read.csv("raw_dataset.csv")
View(raw_data)
head(raw_data)
\end{codeblock}

\begin{codeoutput}{RAW DATA HEAD}
Pregnancies Glucose BloodPressure SkinThickness Insulin 
        6     148            72            35       0 
        1      85            66            29       0 
        8     183            64             0       0 
        1      89            66            23      94 
        0     137            40            35     168 
        5     116            74             0       0 

BMI  DiabetesPedigreeFunction  Age   Outcome
33.6                    0.627  50       1
26.6                    0.351  31       0
23.3                    0.672  32       1
28.1                    0.167  21       0
43.1                    2.288  33       1
25.6                    0.201  30       0
\end{codeoutput}


\section{Data Cleaning}

Real-world medical data is rarely perfect. We must first address biologically 
impossible zero values that represent missing data to avoid biased results.

\subsection{Converting Zeroes Values to NA}

The summary confirms substantial missingness disguised as zeros. 

\begin{codeblock}{R}{R | GETTING COLUMNS WITH ZERO VALUES}
as.matrix(colSums(raw_data == 0))
\end{codeblock}

\begin{codeoutput}{COLUMNS AND NUMBER OF ZERO VALES}
Pregnancies               111
Glucose                     5
BloodPressure              35
SkinThickness             227
Insulin                   374
BMI                        11
DiabetesPedigreeFunction    0
Age                         0
Outcome                   500
\end{codeoutput}

\begin{codeblock}{R}{R | CONVERTING ZERO VALUES TO NA}
col_w_zeroes <- c("Pregnancies", "Glucose", "BloodPressure", "SkinThickness",
                  "Insulin", "BMI")
raw_data[col_w_zeroes] <- lapply(raw_data[col_w_zeroes], function(x) 
                                ifelse(x == 0, NA, x))

summary(raw_data)
\end{codeblock}

\begin{codeoutput}{VISUALIZING SUMMARY ACCOUNTING NA's}
 Pregnancies        Glucose      BloodPressure    SkinThickness      Insulin       
Min.   : 1.000   Min.   : 44.0   Min.   : 24.00   Min.   : 7.00   Min.   : 14.00   
1st Qu.: 2.000   1st Qu.: 99.0   1st Qu.: 64.00   1st Qu.:22.00   1st Qu.: 76.25     
Median : 4.000   Median :117.0   Median : 72.00   Median :29.00   Median :125.00   
Mean   : 4.495   Mean   :121.7   Mean   : 72.41   Mean   :29.15   Mean   :155.55   
3rd Qu.: 7.000   3rd Qu.:141.0   3rd Qu.: 80.00   3rd Qu.:36.00   3rd Qu.:190.00    
Max.   :17.000   Max.   :199.0   Max.   :122.00   Max.   :99.00   Max.   :846.00   
NA's   :111      NA's   :5       NA's   :35       NA's   :227     NA's   :374          

 BMI        DiabetesPedigreeFunction      Age           Outcome     
Min.   :18.20   Min.   :0.0780           Min.   :21.00   Min.   :0.000
1st Qu.:27.50   1st Qu.:0.2437           1st Qu.:24.00   1st Qu.:0.000 
Median :32.30   Median :0.3725           Median :29.00   Median :0.000
Mean   :32.46   Mean   :0.4719           Mean   :33.24   Mean   :0.349 
3rd Qu.:36.60   3rd Qu.:0.6262           3rd Qu.:41.00   3rd Qu.:1.000
Max.   :67.10   Max.   :2.4200           Max.   :81.00   Max.   :1.000 
NA's   :11        
\end{codeoutput}

\vspace{8pt}
Insulin has the highest missing rate, with 374 zero entries. Variables like 
Glucose, Blood Pressure, and BMI also contain invalid zeros that require correction.

\subsection{MICE Imputation}

\vspace{8pt}
Dropping rows with missing data would lose too much information. We will use 
Multivariate Imputation by Chained Equations (MICE) with predictive mean matching 
to estimate and fill these missing values.

\begin{codeblock}{R}{R | MICE IMPUTATION PMM as METHOD}
init <- mice(raw_data, maxit = 0)
methods <- init$method #$
methods[col_w_zeroes] <- "pmm"
imputed_data <- mice(raw_data, method = methods, m = 5, maxit = 5, seed = 123)
final_data <- complete(imputed_data, 1)
head(final_data) 
\end{codeblock}

View full \hyperref[sec:appendix]{R script output here}.

\begin{codeblock}{R}{R | CHECK FINAL DATA CONTAINS NA's}
anyNA(final_data)
\end{codeblock}

\begin{codeoutput}{RETURN TRUE CONTAINS NA's OTHERWISE NONE}
[1] FALSE 
\end{codeoutput}

\subsection{Imputation Validation}

We must verify that our statistical imputation did not distort the natural distribution 
of the data. Visual inspection confirms the integrity of the data. The imputed 
distributions (blue) closely mirror the original distributions (red). This suggests 
the MICE algorithm preserved the underlying data structure without introducing bias.

\vspace{8pt}

Refer to \hyperref[sec:appendix]{Appendix Section} for the full R script.

\begin{figure}[H]
    \centering
    \includegraphics[width=0.8\textwidth]{pregnancies_comparison}
    \captionsetup{style=mycustom}
    \caption{Before and After Imputation (Pregnancies).}
    \label{fig:standard}
    \vspace{1mm}
    % {\footnotesize\textcolor{docgray}{\textit{Source: Generated by LaTeX}}}
\end{figure}


\begin{figure}[H]
    \centering
    \includegraphics[width=0.8\textwidth]{glucose_comparison}
    \captionsetup{style=mycustom}
    \caption{Before and After Imputation (Glucose).}
    \label{fig:standard}
    \vspace{1mm}
    % {\footnotesize\textcolor{docgray}{\textit{Source: Generated by LaTeX}}}
\end{figure}

\begin{figure}[H]
    \centering
    \includegraphics[width=0.8\textwidth]{bloodpressure_comparison.png}
    \captionsetup{style=mycustom}
    \caption{Before and After Imputation (BloodPressure).}
    \label{fig:standard}
    \vspace{1mm}
    % {\footnotesize\textcolor{docgray}{\textit{Source: Generated by LaTeX}}}
\end{figure}

\begin{figure}[H]
    \centering
    \includegraphics[width=0.8\textwidth]{skinthickness_comparison}
    \captionsetup{style=mycustom}
    \caption{Before and After Imputation (SkinThickness).}
    \label{fig:standard}
    \vspace{1mm}
    % {\footnotesize\textcolor{docgray}{\textit{Source: Generated by LaTeX}}}
\end{figure}

\begin{figure}[H]
    \centering
    \includegraphics[width=0.8\textwidth]{insulin_comparison}
    \captionsetup{style=mycustom}
    \caption{Before and After Imputation (Insulin).}
    \label{fig:standard}
    \vspace{1mm}
    % {\footnotesize\textcolor{docgray}{\textit{Source: Generated by LaTeX}}}
\end{figure}

\begin{figure}[H]
    \centering
    \includegraphics[width=0.8\textwidth]{bmi_comparison}
    \captionsetup{style=mycustom}
    \caption{Before and After Imputation (BMI).}
    \label{fig:standard}
    \vspace{1mm}
    % {\footnotesize\textcolor{docgray}{\textit{Source: Generated by LaTeX}}}
\end{figure}

\section{Exploratory Analysis}

Before modeling, we examine the relationships between variables to identify 
potential predictors. We specifically check for multicollinearity to ensure 
our regression models remain stable and interpretable.

\subsection{Correlation Matrix}

The matrix reveals strong correlations between \inlinecode{Glucose} and the diabetes \inlinecode{Outcome}. 
We also observe multicollinearity between \inlinecode{Age} and \inlinecode{Pregnancies}. This validates 
our selection of predictors but suggests caution regarding variance inflation 
in the models.

\begin{figure}[H]
    \centering
    \includegraphics[width=0.8\textwidth]{correlation}
    \captionsetup{style=mycustom}
    \caption{Correlation Matrix Plot}
    \label{fig:standard}
    \vspace{1mm}
    % {\footnotesize\textcolor{docgray}{\textit{Source: Generated by LaTeX}}}
\end{figure}

\begin{codeblock}{R}{R | CORRELATION MATRIX PLOT}
cor_matrix <- cor(final_data, use = "complete.obs")
corrplot(cor_matrix, method = "circle", type = "upper", order = "hclust",
        tl.col = "black", tl.srt = 45, title = "Correlations", mar=c(0,0,1,0))
\end{codeblock}

\subsection{Analyzing Linearity}

We test the linearity assumption for our continuous regression model by examining 
the relationship between Age and Insulin. The local regression plot shows a distinct 
curve. Insulin levels rise in early adulthood but plateau after age 50. A simple 
linear line cannot capture this, justifying the need for non-linear modeling techniques.

\section{Modelling Logistic Regression}

We fit a logistic regression model to predict the binary diabetes \inlinecode{Outcome} using our 
cleaned dataset. The model separates diabetic and non-diabetic patients based on the 
selected risk factors.

\subsection{Model Specification}

\inlinecode{Glucose}, \inlinecode{BMI}, and \inlinecode{Pregnancies} are highly significant predictors 
($p<0.001$). However, \inlinecode{Age} is not statistically significant ($p=0.337$). 
This is likely due to its strong correlation with \inlinecode{Pregnancies}, which 
absorbed the predictive power in this model.

\begin{codeblock}{R}{R | MODELLING LOGISTIC REGRESSION}
set.seed(123)
sample_index <- sample(1:nrow(final_data), 0.8 * nrow(final_data))
train_data <- final_data[sample_index, ]
test_data <- final_data[-sample_index, ]

log_model <- glm(Outcome ~ Glucose + BMI + Age + Pregnancies + 
                DiabetesPedigreeFunction, data = train_data, 
                family= "binomial")

summary(log_model)
\end{codeblock}

\begin{codeoutput}{SUMMARY LOG-RES MODEL}
glm(formula = Outcome ~ Glucose + BMI + Age + Pregnancies + 
                        DiabetesPedigreeFunction, family = "binomial", 
                        data = train_data)

Coefficients:
                          Estimate Std. Error z value Pr(>|z|)    
(Intercept)              -9.200623   0.809675 -11.363  < 2e-16 ***
Glucose                   0.037413   0.004099   9.128  < 2e-16 ***
BMI                       0.079814   0.016719   4.774 1.81e-06 ***
Age                       0.009594   0.010003   0.959 0.337506    
Pregnancies               0.130485   0.037653   3.465 0.000529 ***
DiabetesPedigreeFunction  0.659625   0.321580   2.051 0.040247 *  
---
Signif. codes:  0 '***' 0.001 '**' 0.01 '*' 0.05 '.' 0.1 ' ' 1

(Dispersion parameter for binomial family taken to be 1)

    Null deviance: 796.42  on 613  degrees of freedom
Residual deviance: 568.82  on 608  degrees of freedom
AIC: 580.82

Number of Fisher Scoring iterations: 5
\end{codeoutput}

\subsection{Odds Ratio Interpretation}

We convert the model coefficients into odds ratios to quantify the biological risk 
in human-readable terms. For every one-unit increase in BMI, the odds of diabetes 
rise by roughly 8.3\%. Glucose shows a similar positive risk. The Diabetes Pedigree 
Function has the highest ratio, nearly doubling the odds for each unit increase.

\begin{codeblock}{R}{R | ODDS RATIO}
as.matrix(exp(coef(log_model)))
\end{codeblock}

\begin{codeoutput}{ODDS RATIO OUTPUT}
(Intercept)              0.0001009765
Glucose                  1.0381215630
BMI                      1.0830855110
Age                      1.0096399484
Pregnancies              1.1393803946
DiabetesPedigreeFunction 1.9340669910
\end{codeoutput}

\subsection{Model Performance}

We assess the model's goodness-of-fit by examining the deviance and AIC values. 
The model significantly reduces deviance from 796 to 568. This drop indicates 
that our selected predictors explain a substantial amount of the variation in 
diabetes onset.

\section{Linear Regression (Predict Insulin)}

We shift to a continuous model to predict Insulin levels based on Glucose, BMI, 
and Age. This analysis aims to capture the factors driving insulin fluctuation.

\begin{codeblock}{R}{R | MODELLING LINEAR REGRESSION}
linear_simple <- lm(Insulin ~ Glucose + BMI + Age, 
                    data = final_data)
summary(linear_simple)
\end{codeblock}

\begin{codeoutput}{LINEAR REGRESS OUTPUT}
Call:
lm(formula = Insulin ~ Glucose + BMI + Age, data = final_data)

Residuals:
    Min      1Q  Median      3Q     Max 
-289.02  -45.91  -14.70   25.01  567.44 

Coefficients:
             Estimate Std. Error t value Pr(>|t|)    
(Intercept) -167.6456    19.1120  -8.772  < 2e-16 ***
Glucose        1.8543     0.1125  16.481  < 2e-16 ***
BMI            2.0651     0.4791   4.310 1.84e-05 ***
Age            0.5693     0.2841   2.004   0.0455 *  
---
Signif. codes:  0 '***' 0.001 '**' 0.01 '*' 0.05 '.' 0.1 ' ' 1

Residual standard error: 89.01 on 764 degrees of freedom
Multiple R-squared:  0.3401,    Adjusted R-squared:  0.3375 
F-statistic: 131.2 on 3 and 764 DF,  p-value: < 2.2e-16
\end{codeoutput}

\paragraph{Statement}

\begin{codeblock}{R}{R | LINEAR SPLINE MODEL}
linear_spline <- lm (Insulin ~ Glucose + BMI + bs(Age, degree = 3), 
                    data = final_data)
summary(linear_spline)
\end{codeblock}

\begin{codeoutput}{LINEAR SPLINE MODEL OUTPUT}
Call:
lm(formula = Insulin ~ Glucose + BMI + bs(Age, degree = 3), data = final_data)

Residuals:
    Min      1Q  Median      3Q     Max 
-299.47  -47.78  -14.42   26.00  554.90 

Coefficients:
                      Estimate Std. Error t value Pr(>|t|)    
(Intercept)          -151.3071    18.2230  -8.303 4.61e-16 ***
Glucose                 1.8612     0.1123  16.571  < 2e-16 ***
BMI                     2.2633     0.4871   4.646 3.98e-06 ***
bs(Age, degree = 3)1  -57.2504    32.8240  -1.744   0.0815 .  
bs(Age, degree = 3)2   80.9994    50.5390   1.603   0.1094    
bs(Age, degree = 3)3    5.4944    62.7432   0.088   0.9302    
---
Signif. codes:  0 '***' 0.001 '**' 0.01 '*' 0.05 '.' 0.1 ' ' 1

Residual standard error: 88.82 on 762 degrees of freedom
Multiple R-squared:  0.3447,    Adjusted R-squared:  0.3404 
F-statistic: 80.16 on 5 and 762 DF,  p-value: < 2.2e-16
\end{codeoutput}

\paragraph{Statement}

\subsection{B-Spline Justification}

Standard linear regression assumes a constant rate of change. Our EDA 
showed that aging effects saturate over time. Therefore, we apply a 
cubic B-spline to Age to model this biological non-linearity.

\vspace{8pt}
Refer to \hyperref[sec:appendix]{Appendix Section} for the full R script.

\begin{figure}[H]
    \centering
    \includegraphics[width=0.8\textwidth]{linearity}
    \captionsetup{style=mycustom}
    \caption{B-Spline Plot.}
    \label{fig:standard}
    \vspace{1mm}
    % {\footnotesize\textcolor{docgray}{\textit{Source: Generated by LaTeX}}}
\end{figure}

\subsection{Comparison Linear vs. B-Spline}

We statistically compare the simple linear model against the B-spline model 
using an ANOVA test. The ANOVA results show a p-value of $0.07$. While strictly 
above the $0.05$ threshold, it suggests marginal significance. Given the visual 
evidence of curvature, we retain the spline model for its biological realism.

\begin{codeblock}{R}{R | LINEAR \& B-SPLIME COMPARISON}
linear_simple <- lm(Insulin ~ Glucose + BMI + Age, data = final_data)    
linear_spline <- lm (Insulin ~ Glucose + BMI + bs(Age, degree = 3), data = final_data)
anova(linear_simple, linear_spline)
\end{codeblock}

\begin{codeoutput}{ANOVA OTUPUT}
Analysis of Variance Table

Model 1: Insulin ~ Glucose + BMI + Age
Model 2: Insulin ~ Glucose + BMI + bs(Age, degree = 3)
  Res.Df     RSS Df Sum of Sq      F  Pr(>F)  
1    764 6052997                              
2    762 6010880  2     42116 2.6696 0.06993 .
---
Signif. codes:  0 '***' 0.001 '**' 0.01 '*' 0.05 '.' 0.1 ' ' 1
\end{codeoutput}

\paragraph{Statement}

\subsection{Residual Diagnostics}

We examine the residuals to ensure the model assumptions of homoscedasticity and 
normality are met. Diagnostic plots show that the residuals are randomly scattered, 
satisfying the linearity assumption. Point 460 is flagged as influential but remains 
within acceptable Cook's distance limits. The model fits the data well.

\begin{figure}[H]
    \centering
    \includegraphics[width=0.8\textwidth]{splien_model}
    \captionsetup{style=mycustom}
    \caption{Residual Plot.}
    \label{fig:standard}
    \vspace{1mm}
    % {\footnotesize\textcolor{docgray}{\textit{Source: Generated by LaTeX}}}
\end{figure}

\begin{codeblock}{R}{R | RESIDUAL PLOT}
image_rend <- check_model(linear_spline)
png("splien_model.png", width = 12, height = 9, units = "in", res = 300)
print(image_rend)
dev.off()
\end{codeblock}

\paragraph{Statement}

\section{Discussions}

We synthesize our findings to validate our initial hypotheses. This 
discussion connects our statistical outputs back to the medical literature 
and the project objectives.

\subsection{Validation of Literature Claims}

Our results confirm the claims of Nassiwa and Zeng. \inlinecode{Glucose} and \inlinecode{BMI} were the 
strongest predictors in our logistic model ($\mathbf{p<2e-16}$). This proves that 
immediate metabolic markers are more predictive than general demographics like age.

\subsection{Key Findings}

This study produced three major statistical insights. We successfully imputed 
missing data without bias. Logistic regression identified Glucose, BMI, 
and Pregnancies as primary risk factors. Finally, we demonstrated that the 
relationship between Age and Insulin is non-linear, requiring spline modeling.

\subsection{Limitations (if any)}

The dataset has inherent constraints despite our cleaning efforts. The dataset 
s relatively small, with only \textbf{768} observations. Despite robust imputation, the 
original missingness in Insulin was high. Future studies would benefit from a 
larger, more complete dataset to improve precision.

\section{Appendix}
\label{sec:appendix}

\subsection*{Appendix Imputation Comparison}
\label{sec:appendixa} 
\begin{codeblock}{R}{HISTOGRAM COMPARISON CODE EXAMPLE}
p1 <- gghistogram(raw_data, x = "Pregnancies",
    title = "Original Pregnancies (Raw)",
    xlab = "Pregnancies", ylab = "Count",
    fill = "#b2182b",        
    color = "#b2182b",
    add = "mean", rug = TRUE, add_density = TRUE,
    ggtheme = theme_pubr()
)

p2 <- gghistogram(final_data, x = "Pregnancies",
    title = "Imputed Pregnancies (Complete)",
    xlab = "Pregnancies", ylab = "Count",
    fill = "#2166ac",       
    color = "#2166ac",
    add = "mean", rug = TRUE, add_density = TRUE,
    ggtheme = theme_pubr()
)

ggarrange(p1, p2, ncol = 2, nrow = 1)
\end{codeblock}

\begin{codeblock}{R}{STATISTICAL MODELING}
p1 <- gghistogram(raw_data, x = "Glucose",
    title = "Original Glucose (Raw)",
    xlab = "Glucose", ylab = "Count",
    fill = "#b2182b",        
    color = "#b2182b",
    add = "mean", rug = TRUE, add_density = TRUE,
    ggtheme = theme_pubr()
)

p2 <- gghistogram(final_data, x = "Glucose",
    title = "Imputed Glucose (Complete)",
    xlab = "Glucose", ylab = "Count",
    fill = "#2166ac",       
    color = "#2166ac",
    add = "mean", rug = TRUE, add_density = TRUE,
    ggtheme = theme_pubr()
)

ggarrange(p1, p2, ncol = 2, nrow = 1)
\end{codeblock}

\begin{codeblock}{R}{STATISTICAL MODELING}
p1 <- gghistogram(raw_data, x = "BloodPressure",
    title = "Original BloodPressure (Raw)",
    xlab = "BloodPressure", ylab = "Count",
    fill = "#b2182b",        
    color = "#b2182b",
    add = "mean", rug = TRUE, add_density = TRUE,
    ggtheme = theme_pubr()
)

p2 <- gghistogram(final_data, x = "BloodPressure",
    title = "Imputed BloodPressure (Complete)",
    xlab = "BloodPressure", ylab = "Count",
    fill = "#2166ac",       
    color = "#2166ac",
    add = "mean", rug = TRUE, add_density = TRUE,
    ggtheme = theme_pubr()
)

ggarrange(p1, p2, ncol = 2, nrow = 1)
\end{codeblock}

\begin{codeblock}{R}{STATISTICAL MODELING}
p1 <- gghistogram(raw_data, x = "SkinThickness",
    title = "Original SkinThickness (Raw)",
    xlab = "SkinThickness", ylab = "Count",
    fill = "#b2182b",        
    color = "#b2182b",
    add = "mean", rug = TRUE, add_density = TRUE,
    ggtheme = theme_pubr()
)

p2 <- gghistogram(final_data, x = "SkinThickness",
    title = "Imputed SkinThickness (Complete)",
    xlab = "SkinThickness", ylab = "Count",
    fill = "#2166ac",       
    color = "#2166ac",
    add = "mean", rug = TRUE, add_density = TRUE,
    ggtheme = theme_pubr()
)

ggarrange(p1, p2, ncol = 2, nrow = 1)
\end{codeblock}

\begin{codeblock}{R}{STATISTICAL MODELING}
p1 <- gghistogram(raw_data, x = "Insulin",
    title = "Original Insulin (Raw)",
    xlab = "Insulin", ylab = "Count",
    fill = "#b2182b",        
    color = "#b2182b",
    add = "mean", rug = TRUE, add_density = TRUE,
    ggtheme = theme_pubr()
)

p2 <- gghistogram(final_data, x = "Insulin",
    title = "Imputed Insulin (Complete)",
    xlab = "Insulin", ylab = "Count",
    fill = "#2166ac",       
    color = "#2166ac",
    add = "mean", rug = TRUE, add_density = TRUE,
    ggtheme = theme_pubr()
)

ggarrange(p1, p2, ncol = 2, nrow = 1)
\end{codeblock}

\begin{codeblock}{R}{STATISTICAL MODELING}
p1 <- gghistogram(raw_data, x = "BMI",
    title = "Original BMI (Raw)",
    xlab = "BMI", ylab = "Count",
    fill = "#b2182b",        
    color = "#b2182b",
    add = "mean", rug = TRUE, add_density = TRUE,
    ggtheme = theme_pubr()
)

p2 <- gghistogram(final_data, x = "BMI",
    title = "Imputed BMI (Complete)",
    xlab = "BMI", ylab = "Count",
    fill = "#2166ac",       
    color = "#2166ac",
    add = "mean", rug = TRUE, add_density = TRUE,
    ggtheme = theme_pubr()
)

ggarrange(p1, p2, ncol = 2, nrow = 1)
\end{codeblock}

\subsection{Appendix B-Spline}
\label{sec:appendixb} 
\begin{codeblock}{R}{STATISTICAL MODELING}
ggplot(final_data, aes(x = Age, y = Insulin)) +
  geom_point(color = "#b2182b", alpha = 0.6, size = 2) +
  geom_smooth(method = "lm", 
              formula = y ~ bs(x, degree = 3), 
              color = "#2166ac",   
              fill = "#2166ac",      
              alpha = 0.2,           
              size = 1.5) +          

  labs(title = "B-Spline",
       subtitle = "Non-Linear Relationship between Age and Insulin",
       x = "Age (Years)",
       y = "Insulin Level (mu U/ml)") +
  
  theme_minimal() +
  theme(
    plot.title = element_text(face = "bold", hjust = 0.5, size = 18),
    plot.subtitle = element_text(hjust = 0.5, color = "gray30", size = 14),
    axis.title = element_text(face = "bold", size = 14),
    axis.text = element_text(size = 12))
\end{codeblock}
